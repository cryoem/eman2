%\documentclass[10pt]{article}
%\usepackage{fullpage, graphicx, url}
%\setlength{\parskip}{1ex}
%\setlength{\parindent}{0ex}
%\title{EMAN2 Coding Style}
%\begin{document}

\section {Coding Style}
 \label{CODING-STYLE} 
 \index{Developer's Guide!Coding Style}

\begin{enumerate}
\item Introduction
\begin{itemize}
\item This document summarizes the basic coding and naming style in EMAN2. The intended audiences are EMAN2 developers.

\item This document is not a programming tutorial. You may refer the references at the end of document for good books.


\end{itemize}
\item Overview

\begin{enumerate}
  \item Coding:
    \begin{enumerate}
      \item EMAN2 follows the GNU coding style with minor changes. We use indent at \href{http://www.gnu.org/software/indent/indent.html}{http://www.gnu.org/software/indent/indent.html} for beautifying EMAN2 code. 
    \end{enumerate}
  \item Naming:
    \begin{enumerate}
       \item All source code files use lower cases.
       \item All classes and data types use uppercase in the first letter; 
       \item All functions use lower cases with '\_' if necessary.
     \end{enumerate}
\end{enumerate}

 
\item indent HowTo
\begin{enumerate}

\item  install indent. (for linux, rpm is available from standard distribution.) 
\item save file .indent.pro to your home directory.
\item say you have a file called ``foo.C'', run indent like this: indent foo.C

\item because indent is designed for C code, it is not perfect for C++ code. Read your new source and fix the following possible errors:

%\end{enumerate}
\begin{enumerate}
%\begin{enumerate}
%\begin{enumerate}
\item change 'const const' to 'const'

%\end{enumerate}

\end{enumerate}

\end{enumerate}


 
\item Comments \index{Developer's Guide!Doxygen}

 Use \textbf{Doxygen}
 JavaDoc style: \\ 
%{\color[named]{BrickRed}
\begin{verbatim}
/** Brief description which ends at this dot. Details follow here.\\ 
 */\\ 
class Test\\ 
{\\ 
public:\\ 
	/** The constructor's brief description in one line.\\ 
	 * A more elaborate description of the constructor.\\ 
	 */\\ 
	Test();\\ 
\\ 
	/** do some test.\\ 
	 * @author Liwei Peng <lpeng@bcm.tmc.edu>\\ 
	 * @date 1/20/2005\\ 
	 * @param low the low threshold.\\ 
	 * @param high the high threshold.\\ 
	 * @return 0 if do_test succeeds; 1 if do_test fails.\\ 
	 */\\ 
	int do_test(float low, float high);\\ 
\\ 
	/** Calculate the sum of an array.\\ 
	 * @param[in] data Data array\\ 
	 * @param[in] nitems Number of items in the array.\\ 
	 * @param[out] sum The sum of the array.\\ 
	 */\\ 
	void calc_sum(int[] data, int nitems, int *sum);\\ 
	\\ 
}\\ 
\end{verbatim}%}

\item Samples
\begin{enumerate}
\item \href{./html/emdata_8h-source.html}{emdata.h}
\item \href{./html/emdata_8cpp-source.html}{emdata.cpp}

\end{enumerate}

\item References

\end{enumerate}
\begin{itemize}
\item Doxygen: \url{http://www.stack.nl/~dimitri/doxygen/}


\end{itemize}





Last modifiedby Liwei Peng (lpeng@bcm.tmc.edu)\\ 
\\ 

%\end{document}
