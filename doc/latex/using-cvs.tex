\section{CVS Help} \label{CVS-HELP} \index{CVS Help}
CVS is a version control system. Using it, you can record the history of sources files, and documents. For detailed information, you may refer \href{http://www.gnu.org/software/cvs/}{http://www.gnu.org/software/cvs/}.

This document gives the typical usage of CVS under Unix/Linux
platforms. The intended audience are EMAN2 developers.

\begin{itemize}
  \item Before you use CVS, set up environmental variable CVSROOT. For
  EMAN/EMAN2, set up the following in your shell startup script:
  \begin{itemize}
    \item for csh/tcsh
      \begin{itemize} 
	\item[\%] setenv CVS\_RSH ssh
	\item[\%] setenv CVSROOT
	"blake.3dem.bioch.bcm.tmc.edu:/usr/local/CVS/CVS"
      \end{itemize}
    \item for bash/sh/zsh
      \begin{itemize} 
	\item[\%] export CVS\_RSH=ssh
	\item[\%] export
	CVSROOT="blake.3dem.bioch.bcm.tmc.edu:/usr/local/CVS/CVS"
      \end{itemize}
  \end{itemize}
  NOTE: The following supposes you put EMAN2 under \$HOME/EMAN2.

  \item  To check out EMAN2 source code, run
    \begin{itemize} 
        \item[\%] cd \$HOME/EMAN2/src
        \item[\%] cvs co eman2
    \end{itemize}
  \item To add new files, run
    \begin{itemize} 
      \item[\%] cd \$HOME/EMAN2/src/eman2
      \item[\%] cd YOUR-DIRECTORY
      \item[\%] cvs add YOUR-FILES
      \item[\%] cvs commit
    \end{itemize}
	
  \item To remove files, run
    \begin{itemize} 
      \item[\%] cd \$HOME/EMAN2/src/eman2
      \item[\%] cd YOUR-DIRECTORY
      \item[\%] cvs remove YOUR-FILES
      \item[\%] cvs commit
    \end{itemize}

  \item To check in modified existing files, run
    \begin{itemize} 
      \item[\%] cd \$HOME/EMAN2/src/eman2
      \item[\%] cvs ci
    \end{itemize}

  \item To update your source code to the latest version in the CVS tree, run
    \begin{itemize} 
      \item[\%] cd \$HOME/EMAN2/src/eman2
      \item[\%] cvs update
    \end{itemize}
     NOTE: read the output from the above command. If you see merging conflicts, you must resolve them first.


\end{itemize}