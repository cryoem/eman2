\section{Miscellaneous} \index{Misc.} \label{MISC-DEVELOPERS-GUIDE}

  \subsection{Reading/Writing Images in Python}
    \index{Misc.!Reading/Writing Images}


    Example: Converting from IMAGIC to MRC or SPIDER:

    \begin{verbatim}from EMAN2 import *
e = EMData()
e.read_image(\u201cproj.img\u201d)
e.write_image(\u201cproj.mrc\u201d)     
    # output image format is determined by file extension
e.write_image(\u201cproj.spi\u201d, 1, SPIDER)  
    # explicitly specifiy the output image format   \end{verbatim}
    \normalcolor

    \subsection{Processors Usage} \index{Processors!Usage}
    
    Example: Using Processors in Python

    \begin{verbatim}from EMAN2 import
      
e = EMData()
e.read_image("test1.mrc")
e.process("math.sqrt")
e.process("threshold.binaryrange", {"low" : 5.2, "high" : 10})
e.write_image("output.mrc")     \end{verbatim}
    \normalcolor

    \subsection{Reconstructors Usage} \index{Reconstructors!Usage}

    Example: Using Reconstructors in Python

    \begin{verbatim}from EMAN2 import *
import math

e1 = EMData()
e1.read_image(TestUtil.get_debug_image("samesize1.mrc"))

e2 = EMData()
e2.read_image(TestUtil.get_debug_image("samesize2.mrc"))

r = Reconstructors.get("back_projection")
r.set_params({"size":100, "weight":1})
r.setup()
r.insert_slice(e1, Transform(EULER_EMAN, 0,0,0))
r.insert_slice(e2, Transform(EULER_EMAN, math.pi/2,0,0))

result = r.finish()
result.write_image("reconstructor.mrc")\end{verbatim}
    \normalcolor

    \subsection{Using Aligner, Comparators, and Projectors}
    \index{Aligners!Usage} \index{CMP!Usage} \index{Projectors!Usage}
    Using any of these objects is very similar to the process described in
    Reconstructor Usage above. The general process is
    \begin{verbatim}
obj = CLASS.get(OBJECT_TYPE, {arg1:value1,
  arg2:value2, ...})
data = obj.OBJECT_COMMAND(args)
\end{verbatim}
    Use the EMAN2 html documentation or use the ``dump'' command in
    python to see a list of the available object types for each class
    and their corresponding argument lists.

    \subsection{Using Pyste} \index{Misc.!Pyste}
    EMAN2 uses Pyste to automatically parse C++ code to generate boost
    python wrappers. To use Pyste:

    \begin{enumerate}
    \item
      Install Pyste libraries/tools: 
      \begin{enumerate}
      \item
	Pyste in boost library
      \item
	elementtree 
      \item
	gccxml
      \end{enumerate}
    \item
      Create or modify the pyste file (e.g.,
      eman2/libpyEM/processor.pyste). For a function that return a
      pointer, a return-policy must be defined in the pyste
      file. The typical cases are: 
      \begin{enumerate}
      \item If the function returns a pointer allocated in this
	function, do: \\	    
	{ set\_policy(YOUR\_FUNCTION, return\_value\_policy(manage\_new\_object))}
      \item If the function returns a static pointer , do: \\
	{ set\_policy(YOUR\_FUNCTION, return\_value\_policy(reference\_existing\_object))}
      \item For other cases, do: \\
	{set\_policy(YOUR\_FUNCTION, return\_internal\_reference())}
      \end{enumerate}
    \item
      Run script: eman2/libpyEM/create\_boost\_python
    \end{enumerate}
    
    \subsection{Using FFTW} \index{Misc.!FFTW} \index{FFTW!Usage}

    EMAN2 works with both fftw2 and fftw3. A user makes the choice at
    compile time.  A standard interface is defined to do fft:
    
    {
    \begin{verbatim}class EMfft {
public:
    static int real_to_complex_1d(float *real_data, float *complex_data, 
                                  int n);
    static int complex_to_real_1d(float *complex_data, float *real_data,
                                  int n);
    static int real_to_complex_nd(float *real_data, float *complex_data, 
                                  int nx, int ny, int nz);
    static int complex_to_real_nd(float *complex_data, float *real_data, 
                                  int nx, int ny, int nz);
 };\end{verbatim}
    }
 
    \subsection{Large File I/O} \index{Misc.!Large File I/O}
    \begin{enumerate}
      \item 
	\textbf{portable\_fseek()} should be used for fseek.
      \item
	\textbf{portable\_ftell()} should be used for ftell.
    \end{enumerate}

    \subsection{Euler Angles} \index{Misc.!Euler Angles}
    \begin{itemize}
      \item
	Euler angles are implemented in \textbf{Rotation} class.
      \item
	{
	\begin{verbatim}Rotation r = Rotation(alt, az, phi, Rotation::EMAN);
float alt2 = r.eman_alt();
float az2 = r.eman_az();
float phi2 = r.eman_phi();

float theta = r.mrc_theta();
float phi = r.mrc_phi();
float omega = r.mrc_omega();\end{verbatim}}
    \end{itemize}

    \subsection{Using numpy} \index{numpy!Usage}
    \begin{itemize}
      \item
	In EMAN2,  Numeric array and the corresponding EMData object shares the same memory block.
      \item
	Example: Converting EMData object to numpy array
	{
	\begin {verbatim}from EMAN2 import *
e = EMData()
e.read_image("e.mrc"))
array = EMNumPy.em2numpy(e)\end{verbatim}}
      \item
	Example: Converting Numerc numpy array to EMData object
	{
	\begin{verbatim}from EMAN2 import *
import numpy
n= 100
numbers= range(2*n*n)
array= numpy.reshape(numpy.ndarray(numbers,numpy.Float32),(2*n,n))
e = Wrapper.numpy2em(array)
e.write_image("numpy.mrc")\end{verbatim}}
    \end{itemize}

    \subsection{Using Transformations} \index{Transformations}
    Transform defines a transformation, which can be rotation,
    translation, scale, and their combinations.
    
    Internally a transformation is stored in a 4x3 matrix.
    \( \left[ \begin{array}{ccc}
        a&b&c\\
        e&f&g\\
        i&j&k\\
         m&n&o\\
      \end{array} \right] \)
    
    The left-top 3x3 submatrix
    \(\left[ \begin{array}{ccc}
        a& b& c\\
         e& f& g\\
         i& j& k\\
      \end{array} \right] \) 
    provides rotation, scaling and skewing.
    
    Post translation is stored in 
      \( \left[ \begin{array}{ccc}
	  m&n&o\\
	\end{array} \right] \)
      
      A separate vector containing the pretranslation, with an implicit
      column   \( \left[ \begin{array}{c}
	  0\\0\\0\\1\\
	\end{array} \right] \)at the end when 4x4 multiplies are required.
      
      The 'center of rotation' is NOT implemented as a separate vector,
      but as a combination of pre and post translations.
      
      
      \subsection{Printing Error/Warning/Debugging Information} 
      \index{Debugging!Printing Error Information}
	   \begin{itemize}
	    \item
	      Using the Log Class
	     \begin{enumerate}
	       \item
		 In your main() file, set log level:  Log::logger()
		 \(\longrightarrow\)set\_log\_level(WARNING\_LOG); 
	       \item
		 Log message in different levels: (log functions use the same
		 argument format like printf()).
		 \begin{itemize}
		   \item[] LOGERR("out of memory");
		   \item[] LOGWARN("invalid image size");
		   \item[] LOGDEBUG("image mean density =
		   \%f\(\backslash n\)",
		   mean);
		   \item[] LOGVAR("image size = (\%d, \%d,
		   \%d)\(\backslash n\)", nx, ny, nz);
		 \end{itemize}
	       \item
		 To log function enter point, use ENTERFUNC; To log
		 function exit point, use EXITFUNC. 
	     \end{enumerate}
	   \end{itemize} 

     \subsection{Adding Testing Groups} \index{Testing Groups}
      \begin{itemize}
	 \item
	   These group tags are already defined in file, "eman2doc.h":
	   \begin{itemize}
	     \item tested0 : code not yet complete
	     \item tested1 : code complete but untested
	     \item tested2 : code complete but contains bugs
	     \item tested3 : tested
	     \item tested3a : manual testing
	     \item tested3b : unit test in C++
	     \item tested3c : unit test in Python
	     \item tested3d : incorporated into sucessful regression test
	   \end{itemize}
 	 \item
 	   How to use these tag to label testing group:
 	   \begin{itemize}
	     \item add /**@ingroup tested3c*/ to the beginning of a
	     class tested in Python, then the corresponding class will
	     be labeled "unit test in Python" in doxygen generated
	     document. 
             \item  you can also define other grouping tag, just
             follow the testing group example in "eman2doc.h" 
              \item a single unit can be labeled for multiple group 
	   \end{itemize}
 	   \end{itemize}

	   
